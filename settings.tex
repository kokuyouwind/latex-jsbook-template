\usepackage[dvipdfmx]{graphicx}
\usepackage{amsmath}
\usepackage{euler}
\usepackage{okumacro}
\usepackage{tikz}
\usepackage{wrapfig}
\usepackage{setspace}
\setstretch{0.95} % ページ全体の行間を設定
% listings
\usepackage{listings,jlisting}
\definecolor{OliveGreen}{cmyk}{0.64,0,0.95,0.40}
\definecolor{colFunc}{rgb}{1,0.07,0.54}
\definecolor{CadetBlue}{cmyk}{0.62,0.57,0.23,0}
\definecolor{Brown}{cmyk}{0,0.81,1,0.60}
\definecolor{colID}{rgb}{0.63,0.44,0}

\lstset{
    language=C,%プログラミング言語によって変える。
    basicstyle={\ttfamily\small},
    %keywordstyle={\color{OliveGreen}},
    %[2][3]はプログラミング言語によってあったり、なかったり
    %keywordstyle={[2]\color{colFunc}},
    %keywordstyle={[3]\color{CadetBlue}},%
    %commentstyle={\color{Brown}},
    commentstyle={\ttfamily\small},
    %identifierstyle={\color{colID}},
    %stringstyle=\color{blue},
    tabsize=2,
    frame=single,
    numbers=none,
    %numberstyle={\ttfamily\small},
    breaklines=true,%折り返し
    backgroundcolor={\color[gray]{.95}},
    captionpos=b,
    showstringspaces=false,
    %xleftmargin=10px,
    %xrightmargin=10px
    xleftmargin=\parindent,
    xrightmargin=\parindent,
}
\renewcommand{\lstlistingname}{コード}
% hyperref
\usepackage[dvipdfmx]{hyperref}
\usepackage{url}
\usepackage{atbegshi}
\ifnum 42146=\euc"A4A2
\AtBeginShipoutFirst{\special{pdf:tounicode EUC-UCS2}}
\else
\AtBeginShipoutFirst{\special{pdf:tounicode 90ms-RKSJ-UCS2}}
\fi
\usetikzlibrary{shapes,backgrounds}
%\usepackage{makeidx}
%\makeindex
\setlength{\textwidth}{\fullwidth}
\setlength{\evensidemargin}{\oddsidemargin}
%デフォルトをゴシックにする
\renewcommand{\kanjifamilydefault}{\gtdefault}
%各種コマンド
\newcommand{\urlnote}[1]{\footnote{\url{#1}}}
\def\chapterautorefname~#1\null{第~#1章\null}
\newcommand{\refch}[1]{\autoref{ch:#1}}
\def\sectionautorefname~#1\null{~#1節\null}
\newcommand{\refsec}[1]{\autoref{sec:#1}}
\def\lstlistingautorefname~#1\null{コード~#1\null}
\newcommand{\refcode}[1]{\autoref{code:#1}}
\def\figureautorefname~#1\null{図~#1\null}
\newcommand{\reffig}[1]{\autoref{fig:#1}}

%表紙・裏表紙の挿入コマンド
\newcommand{\includecover}[1]{
  \enlargethispage{\paperwidth}
  \thispagestyle{empty}
  \vspace*{-1truein}
  \vspace*{-\topmargin}
  \vspace*{-1.16\headheight}
  \vspace*{-\headsep}
  \vspace*{-\topskip}
  \noindent\hspace*{-1.04in}\hspace*{-\oddsidemargin}
  \includegraphics[width=1.005\paperwidth]{#1}
  \newpage
}
